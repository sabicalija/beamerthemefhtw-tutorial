\documentclass[aspectratio=169]{beamer}
\usetheme[style=noir]{fhtw}
\usepackage{preamble}

\title[Theme User Guide]{Beamer Theme \glsentrytext{uastw}}
\subtitle{User Guide}
\author{Alija Sabic, \glsentrytext{msc}}
\mail{sabic.alija@gmail.com}
\institute{Department Electronic Engineering}

\begin{document}

\begin{frame}[plain]
    \titlepage
    % \glsadd{uastw}
    % \glsadd{msc}
\end{frame}

\section{Introduction}

\begin{frame}
    \par This tutorial describes how to use the \LaTeX~beamer theme \texttt{fhtw} \cite{sabic:github:beamerthemefhtw} of the \gls{uastw}.
\end{frame}


\section{Configuration}
\subsection{Styles}

\foreach \style in {bridge,complex,fhtw-light,fhtw-random-light,fhtw-random,fhtw-simple,fhtw,noir-bridge-color,noir-bridge,noir,simple}{
        \subsubsection{\style}

        \begin{frame}
            \begin{listing}[H]
                \inputsource[firstline=2,lastline=2]{latex}{\style.tex}
                \caption{Using style \style}
                % \label{lst:title:noir}
            \end{listing}

            \begin{figure}
                \setlength{\fboxsep}{0pt}
                \fbox{\includegraphics[width=0.7\textwidth]{\style}}
                \caption{\style~ style title page.}
                % \label{fig:title:noir}
            \end{figure}

        \end{frame}
    }

% \subsubsection{noir}

% \begin{frame}
%     \begin{listing}[H]
%         \inputsource[firstline=2,lastline=2]{latex}{noir.tex}
%         \caption{Using style noir}
%         \label{lst:title:noir}
%     \end{listing}

%     \begin{figure}
%         \includegraphics[width=0.7\textwidth]{noir}
%         \caption{noir style title page.}
%         \label{fig:title:noir}
%     \end{figure}

% \end{frame}

% \subsubsection{noir-bridge}

% \begin{frame}
%     \begin{listing}[H]
%         \inputsource[firstline=2,lastline=2]{latex}{noir-bridge.tex}
%         \caption{Using style noir-bridge}
%         \label{lst:title:noir-bridge}
%     \end{listing}

%     \begin{figure}
%         \includegraphics[width=0.7\textwidth]{noir-bridge}
%         \caption{noir-bridge style title page.}
%         \label{fig:title:noir-bridge}
%     \end{figure}

% \end{frame}

% \subsubsection{noir-bridge-color}

% \begin{frame}
%     \begin{listing}[H]
%         \inputsource[firstline=2,lastline=2]{latex}{noir-bridge-color.tex}
%         \caption{Using style noir-bridge-color}
%         \label{lst:title:noir-bridge-color}
%     \end{listing}

%     \begin{figure}
%         \includegraphics[width=0.7\textwidth]{noir-bridge-color}
%         \caption{noir-bridge-color style title page.}
%         \label{fig:title:noir-bridge-color}
%     \end{figure}

% \end{frame}

\section{Assets}
\subsection{Figures}

\subsection{Tables}
\subsubsection{Basic}
\begin{frame}
    \begin{table}
        \begin{tblr}{width=\linewidth,colspec={XX[3]},hlines,vlines,row{2}={font=\tt}}
            Option & Value                                                                                                         \\
            style  & fhtw, fhtw-light, fhtw-random, fhtw-random-simple, fhtw-complex, noir, noir-bridge, noir-bridge-color, simple \\
        \end{tblr}
        \caption{Simple table example.}
        \label{tab:example:simple}
    \end{table}
\end{frame}

\subsubsection{Advanced}
\begin{frame}
    \begin{table}
        \begin{tblr}{width=\linewidth,colspec={XXX},hlines,vlines,row{1}={bg=tw-blue,fg=white}}
            Math  & Is  & Fun \\
            Proof & 'em & all \\
        \end{tblr}
        \caption{Advanced table example.}
        \label{tab:example:advanced}
    \end{table}
\end{frame}

\appendix

\begin{frame}[allowframebreaks]{Acronyms}
    \printglossary[type=\acronymtype, nonumberlist]
\end{frame}

\begin{frame}[allowframebreaks]{References}
    \bibliography{references}
\end{frame}

\end{document}